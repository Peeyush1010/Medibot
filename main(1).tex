\documentclass{spie}

\usepackage{cite}

\usepackage{times} % Times New Roman
\usepackage{indentfirst} % Indent first paragraphs
\usepackage{hyperref}
\usepackage{algorithmic}
\usepackage{graphicx}
\usepackage{amsmath}
\usepackage{textcomp}
\usepackage{hyperref}
\usepackage{xcolor}
\usepackage{amssymb}
\usepackage{float}
\usepackage{booktabs}
\usepackage{makecell}
\usepackage{colortbl}
\usepackage{xcolor}
\usepackage{adjustbox}

\usepackage[colorinlistoftodos,color=pink]{todonotes} % For fancy \todo{something}
\usepackage{xkeyval}
\presetkeys{todonotes}{inline}{}

\usepackage{lipsum} % For lorem ipsum, remove this

\newcommand{\fourier}{\mathcal{F}}

\title{Medibot: A helpful assistant in Health care}

\author{Peeyush Raj\supit{1}, Nitish Kumar\supit{1}		
\\
  \normalsize 
  \supit{1} Amity University, Noida, India; \\
}

\begin{document}

\maketitle

\begin{abstract}
This project is about making a medical chatbot named MediBot. It helps people by asking questions about their health problems and giving basic suggestions. MediBot is not a real doctor, but it gives early advice based on the symptoms shared by the user.
We trained MediBot using a small language model called LLaMA 3.2–1B. To make training faster and use less memory, we used a technique called LoRA (Low-Rank Adaptation). We created a dataset with 2,100 conversations between a doctor and a patient. These conversations were divided into three groups: mild, moderate, and serious.
Each conversation includes several questions and answers, a short summary, and a suggestion. The chatbot learns how to ask follow-up questions and give suggestions such as home remedies, over-the-counter medicine, or visiting a doctor.
The training was done using Google Colab, and testing was done on a laptop with an RTX 4060 GPU. A simple chatbot interface was made using Gradio, so users can enter their symptoms and see the reply. This project shows that it is possible to make a helpful medical chatbot using small models and simple tools.
  
  \keywords{}
\end{abstract}

\section{Introduction}
\label{sec:intro}


\section{Related Work}
\label{sec:related}

This paper \cite{tan2024fine} has dicussed.

\section{Methodology}

\section{Experiments}
\subsection{Dataset}
\subsubsection{Implementation Details}
\subsection{Result and Dicussion}


\section{Conclusion}


\bibliographystyle{spiebib}
\bibliography{bibliography}
,

%%%%%%%%%%%%%%%%%%%%%%%%%%%%%%%%%%%%%%%%%%%%%%%%%%%%%%%%%%%%%%%%%%%%%%%%%%%%
% Author information table (place right after the references)
%%%%%%%%%%%%%%%%%%%%%%%%%%%%%%%%%%%%%%%%%%%%%%%%%%%%%%%%%%%%%%%%%%%%%%%%%%%%
%%%%%%%%%%%%%%%%%%%%%%%%%%%%%%%%%%%%%%%%%%%%%%%%%%%%%%%%%%%%%%%%%%%%%%%%%%%%



\clearpage





\end{document}
